\chapter{Intelligenztests}
Testen die Fähigkeiten in einem Bereich, ermitteln g (g-Loading)
\section{Frühe Versuche}
\section{Binet-Simon-Skala}
Als erster \glqq seriöser\grqq{} Intelligenztest kann die Binet-Simon-Skala angesehen werden. Sie wurde vom französischen Bildungsministerium in Auftrag gegeben um eine objektive Einteilung von Kindern auf eine Sonderschule zu ermöglichen. Zuvor erfolgte dies durch eine Einschätzung des Lehrers, was oft zu Problemen geführt hat. Entwickelt wurde dieser Intelligenztest von den Franzosen Alfred Binet und Théodore Simon im Jahr 1905.

\paragraph{Aufbau}
Der Test besteht aus vielen Einzelaufgaben, die je nach Schwierigkeit in Alterklassen eingeteilt sind. Ein normal entwickeltes Kind aus diesen Alters sollte die Aufgabe lösen können. Einige Beispiele für diese Aufgaben sind:
\begin{itemize}
  \item{Das Unterscheiden von Rechts und Links}
  \item{Von 20 bis 0 rückwärts zählen}
  \item{Die Wörter eines Satzes in die richtige Reihenfolge bringen}
\end{itemize}

\paragraph{Auswertung}
Die höchste Altersklasse, aus der alle Aufgaben gelöst wurden, ist das Grundalter des Probanden.
Danach alle richtigen Aufgaben aus höheren Altersklasse als Anteile addiert. Dabei ist es egal, aus welcher höheren Altersklasse die Aufgabe gelöst wurde. Eine richtige Aufgabe dem Bereich 8 Jahre zählt genau so viel, wie eine richtige Aufgabe aus dem Bereich 10 Jahre.

Ein Kind hat zum Beispiel folgendes Ergebnis:

\begin{tabular}{l|lll}
  \textbf{Alter} & \textbf{Aufg. 1} & \textbf{Aufg. 2} & \textbf{Aufg. 3} \\
  3              & \cmark           & \cmark           & \cmark           \\
  4              & \cmark           & \cmark           & \cmark           \\
  5              & \xmark           & \cmark           & \xmark           \\
  6              & \xmark           & \xmark           & \cmark           \\
  7              & \xmark           & \xmark           & \xmark           \\
  8              & \xmark           & \xmark           & \xmark           \\
\end{tabular}

Sein geistiges Grundalter beträgt 4 Jahre, da es aus dieser Altersgruppe alle Aufgaben gelöst hat und in allen höheren Altersgruppen mindestens einen Fehler hat. Außerdem sind aus den höheren Altersgruppen noch 4 Aufgaben gelöst worden. Da es pro Gruppe von einem Jahr 3 Aufgaben gibt zählt eine Aufgabe:
$$ \frac{12\,\mathrm{Monate}}{3\,\mathrm{Aufgaben}} = 4\,\mathrm{Monate} $$
Diese Monatsanteile werden dann zum Grundalter addiert:
$$ 4\,\mathrm{Jahre} + 2 \times 4\,\mathrm{Monate} = 4\,\mathrm{Jahre}, 8\,\mathrm{Monate} $$
Dieses ermittelte geistige Alter wird dann mit dem tatsächlichen Lebensalter verglichen.
Wäre das getestete Kind z.B. 4 Jahre alt, dann wäre es überdurchschnittlich weit entwickelt. Würden die Ergebnisse aber
von einem 6-jährigen stammen, währe das Kind in seiner geistigen Entwicklung zurückgeblieben.

\paragraph{Vor- und Nachteile}
Die Binet-Simon-Skala war ein Durchbruch im Vergleich zu vorherigen Methoden. Sie findet (in abgewandelter Form) auch heute noch, über 100 Jahre später, Verwendung in Intelligenztests. In der beschriebenen ursprünglichen Variante hatte sie allerdings einige Probleme. Das größte Problem war, dass der reine Altersunterschied keine vergleichbaren Aussagen ermöglicht. Denn die Entwicklung eines Kindes verläuft nicht gleichmäßig. Kleinkinder entwickeln sich schneller als größere Kinder von z. B. 12 Jahren. Wenn also ein 5-jähriges Kind ein geistiges Alter von 4 Jahren hat, ist dies schlimmer als wenn ein 11-jähriges Kind ein geistiges Alter von 10 Jahren hat, obwohl der Altersunterschied in beiden Fällen 1 Jahr beträgt.

\section{Intelligenzquotient nach William Stern}
Aus dem Problem der Binet-Simon-Skala entwickelte William Stern den Intelligenzwotienten, der die Differenz zwischen Lebensalter und geistigem Alter vergleichbar machte. Dazu bildet er den Quotient aus Lebensalter und geistigem Alter.
$$ \mathrm{IQ} = \frac{\mathrm{Intelligenzalter}}{\mathrm{Lebensalter}} \times 100 $$
Der Quotient wird noch mit 100 multipliziert, um eine einfacher zu handhabende Zahl zu erhalten.

%TODO: Unify terms "geistiges Alter" and "Intelligenzalter"
