\chapter{Fazit}
Intelligenz ist nach wie vor ein sehr heikles und umstrittenes Thema und wird es wohl noch lange bleiben. Doch auch wenn es keine allgemeine Definition von Intelligenz gibt, hat jeder seine eigenen Vorstellungen davon. Beide vorgestellten Intelligenzmodelle haben Befürworter und Gegner.

Intelligenztests spielen eine wichtige Rolle, sowohl in der Wissenschaft als auch in der Forschung. Allerdings sollten ihre Ergebnisse nicht als präzise wissenschaftliche Messung angesehen werden. Selbst nach 100 Jahren wird ihre Aussagekräftigkeit noch diskutiert und Phänomene wie der Flynn-Effekt sind starke Indizien, dass unsere aktuellen Intelligenztests nicht sehr zuverlässig darin sind, Intelligenz zu messen.

Wie Howard Gardner anmerkte, werden Intelligenztests viel zu oft verwendet, um Menschen in Schubladen zu stecken, ohne ihre gesamten Fähigkeiten zu erfassen.

Die Frage \glqq Ist Intelligenz messbar?\grqq{} würde ich also mit \emph{Nein} beantworten.
