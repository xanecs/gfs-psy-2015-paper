\chapter{Der Generalfaktor der Intelligenz}

Der Generalfaktor, auch g-Faktor genannt, ist ein Modell zur Beschreibung der
menschlichen Intelligenz, welches 1904 vom britischen Psychologen
Charles Spearman veröffentlicht wurde. \cite{wiki_spearman}

\section{Charles Spearman}
Charles Spearman wurde am 10. September 1863 in London geboren. \cite{wiki_spearman} Er kam aus wohlhabenden Verhältnissen, besuchte eine Privatschule und erhielt am College eine Ausbildung in Ingenieurswesen. Nachdem er 15 Jahre als Offizier in der britischen Armee tätig war, fasste er den Entschluss experimentelle Psychologie zu studieren, eine Wissenschaft, die ihn insgeheim seit seiner Schulzeit interessierte. \cite{galton_spearman} Aufgrund seines für den Fachbereich etwas ungewöhnlichen Hintergrundes und mangelnden Qualifikationen wählte er die Universität Leipzig aus, die lockere Anforderungen stellte. 1906 erhilet er seinen Doktortitel, war zu diesem Zeitpunkt aber bereits für seine Intelligenztheorie berühmt geworden, die er 1904 unter dem Titel \glqq General Intelligence, objectively determined and measured\grqq{} (zu Deutsch etwa: Generelle Intelligenz, objektiv bestimmt und vermessen) veröffentlichte.
\cite{wiki_en_spearman} \cite{york_spearman}

\section{Zwei-Faktoren-Theorie}


\cite{wiki_intheorie}
