\chapter{Der Generalfaktor der Intelligenz}

Der Generalfaktor, auch g-Faktor genannt, ist ein Modell zur Beschreibung der
menschlichen Intelligenz, welches 1904 vom britischen Psychologen
Charles Spearman veröffentlicht wurde. \cite{wiki_spearman}

\section{Charles Spearman}
Charles Spearman wurde am 10. September 1863 in London geboren. \cite{wiki_spearman} Er kam aus wohlhabenden Verhältnissen, besuchte eine Privatschule und erhielt am College eine Ausbildung in Ingenieurswesen. Nachdem er 15 Jahre als Offizier in der britischen Armee tätig war, fasste er den Entschluss experimentelle Psychologie zu studieren, eine Wissenschaft, die ihn insgeheim seit seiner Schulzeit interessierte. \cite{galton_spearman} Aufgrund seines für den Fachbereich etwas ungewöhnlichen Hintergrundes und mangelnden Qualifikationen wählte er die Universität Leipzig aus, die lockere Anforderungen stellte. 1906 erhilet er seinen Doktortitel, war zu diesem Zeitpunkt aber bereits für seine Intelligenztheorie berühmt geworden, die er 1904 unter dem Titel \glqq General Intelligence, objectively determined and measured\grqq{} (zu Deutsch etwa: Generelle Intelligenz, objektiv bestimmt und vermessen) veröffentlichte.
\cite{wiki_en_spearman} \cite{york_spearman}

\section{Zwei-Faktoren-Theorie}
Spearman untersuchte die Ergebnisse verschiedener Testreihen. Die Tests fragten mehrere Bereiche ab, wie Sprache, Arithmetik(Rechnen) und Logik. Mit diesen Ergebnissen führte er eine Faktoranalyse durch. Bei einer Faktoranalyse stellt man fest, wie stark die Ergebnisse \emph{korrelieren}, das heißt, wie stark sie voneinander abhänge. Er extrahierte dabei mehrere Faktoren: Mehrere s-Faktoren, die zeigen, wie stark verschiedene Bereiche zusammenhängen. So sind z.B. die Aufgaben aus Arithmetik und Logik sehr ähnlich. Es ist wahrscheinlich, dass eine Person, die im Arithmetik-Test gut abschneiden auch im Logik-Test gute Ergebnisse erzielen.

Zusätzlich zu den s-Faktoren extrahierte er noch den g-Faktor. Er verbindet alle Gebiete miteinander und steht somit über den s-Faktoren. Die Ergebnisse aller Bereiche korrellieren miteinander, das heißt, ist eine Person in einem Bereich gut ist es wahrscheinlich, dass sie auch in \emph{allen} anderen Bereichen gut sind. Personen mit Inselbegabung, das heißt, Personen, die in Logik sehr gute Ergebnisse haben, dafür in allen anderen Bereichen druchschnittlich sind, sind die Seltenheit.

\cite{wiki_intheorie}

\section{Kontroverse}
Der Generalfaktor nach Charles Spearman war seit beginn sehr umstritten. Die Korrelationen, die sich durch die Faktoranalyse ergeben sind zwar gegeben, aber nicht sehr hoch. Auch stellt sich die Frage, was Intelligenztests überhaupt messen können (siehe \ref{sec:flynn}~Flynn-Effekt).


\subsection{Neurologie}
Untersuchungen mit Hilfe von Magnetresonanztomographie (MRT), bei denen gezeigt wird, welche Bereiche des Gehirns besonders aktiv sind bestätigen, dass die Intelligenz über mehrere Fachgebiete hinaus denselben Ursprung hat.
Probanden wurden untersucht, während sie Intelligenztests aus verschiedenen Bereichen lösten. Bei allen Aufgaben zeigte dieselbe Hirnregion starke Aktivität: Eine "genau Umrissene Region des Stirnlappens, [der] laterale Präfrontale Kortex" \cite{geok15}.
