\chapter{Multiple Intelligenztheorie}
Eine Alternative Intelligenztheorie zum Generalfaktor der Intelligenz ist die Multiple Intelligenztheorie von Howard Gardner.

\section{Howard Gardner}
Howard Gardner ist ein amerikanischer Psychologe. Er wurde am 11. Juli 1943 in Pennsylvania geboren. Er studierte an der Harvard University in Cambridge, zuerst Jura, schwenkte jedoch auf Psychologie und Pädagogik um. Dort promovierte Gardner 1971 und arbeitet dort noch heute als Professor. \cite{cv_gardner}

\section{Allgemeine Überlegungen}
Howard Gardner kritisierte die Vorstellung, Intelligenz als eine einzige Zahl zu sehen, nach der sich jeder Mensch einteilen lässt. Seiner Ansicht nach reicht diese eindimensionale Skala nicht aus, um die Fähigkeiten eines Individuums zu erfassen.

Er kritisiert auch, dass Intelligenztests dazu missbraucht werden, Kinder schon früh in \glqq Schubladen zu stecken\grqq{}. Kinder, die gut in Logik oder Mathematik sind werden gefördert, während Kinder mit anderen Fähigkeiten wie Kreativität oder Sportlichkeit benachteiligt werden.

Statt also die Intelligenz als eine einzelne Fähigkeit zu sehen, nennt er zuerst sieben (später acht) Teilintelligenzen. \cite{gardner_mi}

\section{Die Intelligenzen}
Nach Gardner gibt es acht Intelligenzen, die bei jedem Menschen unterschiedlich stark ausgeprägt sind. In seiner ursprünglichen Fassung der multiplen Intelligenztheorie beschrieb er nur sieben Formen, später kam die Naturkenntnis dazu.
\paragraph{Logisch-Mathematisch}
Die Fähigkeit mit Zahlen umzugehen, Thesen aufzustellen und diese zu beweisen oder zu widerlegen.
\paragraph{Räumlich}
Sich dreidimensionale Sachverhalte vorstellen und analysieren können.
\paragraph{Interpersonell}
Empathiefähigkeit, die Gefühle und Beweggründe anderer Personen erfassen.
\paragraph{Intrapersonell}
Die eigenen Gefühle und Bedürfnisse verstehen.
\paragraph{Sprachlich}
Sich selbst mit Worten ausdrücken und Äußerungen Anderer verstehen.
\paragraph{Musikalisch}
Taktgefühl, Tonhöhen unterscheiden, musikalische Kreativität.
\paragraph{Körperlich-Kinästhetisch}
Koordination von Bewegungen, Reflexe und das Erfassen von Bewegungen Anderer.
\paragraph{Naturkenntnis}
Die Natur beobachten und verstehen.

\section{Multiple-Intelligenz-Tests}
Es gibt bis heute keinen offiziellen Multiple-Intelligenz-Test, wie solche, die es für den Generalfaktor der Intelligenz gibt. Es gibt allerdings Fragebögen, ähnlich wie Persönlichkeitstests, die die \glqq Stärken\grqq{} eines Menschen bestimmen. Andere MI-Tests beinhalten praktische Übungen, in denen die Testpersonen zeigen, wie gut sie eine Fähigkeit beherrschen, z.B. Tonleitern hören und unterscheiden. \cite{mi_tests}
